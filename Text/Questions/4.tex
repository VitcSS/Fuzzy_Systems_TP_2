\question Podemos definir as funções como :
\[\mu_{very\text{ }young}(x) = \int_{X} \mu_{young}^{2}(x) dx\]
\[\mu_{very\text{ }old}(x) = \int_{X} \mu_{old}^{2}(x) dx\]
As negativas são :
\[\mu_{not\text{ }very\text{ }young}(x) = \int_{X} 1-\mu_{very\text{ }young}(x) dx\]
\[\mu_{not\text{ }very\text{ }old}(x) = \int_{X} 1-\mu_{very\text{ }old}(x) dx\]

\subsection*{Not Very Old AND Not Very Young }
Utilizando a t-norma produto encontramos computacionalmente a seguinte função de pertencimento :
\begin{figure}[ht!]
    \centering
    \includegraphics[scale = 0.5]{images/a.png}
\end{figure}
\subsection*{Very Old and Very Young}
Utilizando a t-norma teremos um resultado irrisório por se tratar de uma relação paradoxal :
\begin{figure}[ht!]
    \centering
    \includegraphics[scale = 0.5]{images/b.png}
\end{figure}