\question 
Regra :
\begin{center}
    \begin{itemize}
        \item Se \(x\) é \(A_1\) então \(y\) é \(C_1\)
        \item Se \(x\) é \(A_2\) então \(y\) é \(C_2\)
    \end{itemize}
\end{center}

Fato : 

\begin{center}
    \begin{itemize}
        \item \(x\) é \(A'\)
    \end{itemize}
\end{center}

Conclusão :

\begin{center}
    \begin{itemize}
        \item \(y\) é \(C'\)
    \end{itemize}
\end{center}
\pagebreak
Processo de racícionio nebuloso :
\begin{enumerate}
    \item Encontrar o máximo da t-norma entre \(A_1\) e \(A'\), obtendo o valor de \(w_1\)
    \begin{figure}[ht!]
        \centering
        \includegraphics[scale=0.5]{images/1.png}
    \end{figure}
    \item Realizar a t-norma entre \(w_1\) e \(C_{1}\), obtendo assim \(C_{1}'\). \(C_{1}'\) segue os valores de \(C_1\) limitados a \(w_1\)
    \begin{figure}[ht!]
        \centering
        \includegraphics[scale=0.5]{images/2.png}
    \end{figure}
    \item Encontrar o valor máximo da t-norma entre \(A_2\) e \(A'\), obtendo o valor de \(w_2\)
    \begin{figure}[ht!]
        \centering
        \includegraphics[scale=0.5]{images/3.png}
    \end{figure}
    \item Realizar a t-norma entre \(w_2\) e \(C_{2}\), obtendo assim \(C_{2}'\). \(C_{2}'\) segue os valores de \(C_2\) limitados a \(w_2\)
    \begin{figure}[ht!]
        \centering
        \includegraphics[scale=0.5]{images/4.png}
    \end{figure}
    \item Realizar s-norma entre \(C_{1}'\) e  \(C_{2}'\), obtendo assim \(C'\)
    \begin{figure}[ht!]
        \centering
        \includegraphics[scale=0.5]{images/5.png}
    \end{figure}
\end{enumerate}

